\documentclass[11pt,titlepage]{article}
\usepackage{textcomp}
\usepackage{fullpage}
\usepackage{amsmath}
\usepackage{amssymb}
\usepackage{gensymb}
\usepackage{color}
\usepackage{graphicx}
\graphicspath{ {images/} }
\usepackage{float}
\restylefloat{table}
\usepackage{array}

\author{Rane Brown \\ Brian Douglass}
\title{ECEN 4593: Memory Simulation Project}
\date{\today}

\begin{document}
\maketitle
\tableofcontents
\listoffigures
\listoftables
\newpage

\section{Introduction}
    This project simulates a two level cache using the C programming language. The memory hierarchy consists of a L1 instruction cache, a L1 data cache and a unifed L2 cache. In addition, each cache level has an associated victim cache which is setup as an eight deep fully associative cache. If there is a miss at a particular cache level then that level's victim cache is checked before making a request to the next level in the hierarchy. The caches are \emph{write-allocate}, \emph{write-back} caches which means a dirty bit must be used to track when a write request occurs. Each cache maintains a LRU (least recently used) replacement policy for each set in a set associative cache or in a fully associative cache. This project evaluates the performance of nine different memory configurations using sample traces from a selection of six SPEC benchmarks. The bus widths, transfer times, hit/miss times, and various other important paramaters can be found in the official project description. 

\section{Setup and Code Structure}

\section{Performance Evaluation}

    \subsection{astar}

    \subsection{bzip2}

    \subsection{gobmk}

    \subsection{libquantum}

    \subsection{omnetpp}

    \subsection{sjeng}

\section{Overall Performance Comparison}

\section{Cost Evaluation}

    \subsection{Cost versus Performance}

\section{Main Memory Bandwidth Increase}

\section{Conclusion}

\end{document}
